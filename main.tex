\documentclass{programmierpraktikum}
\vorlesung{Programmierpraktikum}
\semester{Sommersemester 2012}
\betreuer{Wilfried Linder}
\subtitle{Bomberman}

\begin{document}

\maketitle
Wie in Meilenstein 3 angegeben, müssen für das Bestehen zusätzliche Features implementiert werden. Diese können aus dem Katalog unten ausgewählt werden. Es ist ebenfalls möglich, eigene Features vorzuschlagen. Die Tutoren entscheiden dann von Fall zu Fall, wie viele Punkte vergeben werden.
\section{Erweiterungen Bomberman}

\begin{longtable}{|p{0.75\textwidth}|p{0.25\textwidth}|}
\hline
\textbf{Erweiterung} & \textbf{Punkte} \\ \hline
Weite Karten / Levels & nach Aufwand \\ \hline
Zufallsgenerator für Karten. Mehr Punkte werden vergeben, wenn der Generator z.B. Konsistenzprüfung beherrscht, also testen kann, ob ein Level gültig ist. Das heißt die Ausgänge können erreicht werden und kein Spieler ist bevorteilt & 1 bzw. 3 \\ \hline
Karteneditor, mit dem neue Levels angelegt werden können. Dabei sollen die Startpunkte der Spielfigur(en), Mauerstücke und Ausgänge und so weiter gesetzt werden können. Mehr Punkte werden vergeben, wenn der Editor z.B. Konsistenzprüfung beherrscht, also testen kann, ob ein Level gültig ist. Das heißt die Ausgänge können erreicht werden und kein Spieler ist bevorteilt. & 2 bzw. 5 \\ \hline
Sound. Ausgabe von Explosionseffekten, Hintergrundmusik, etc. & 1 \\ \hline
Applet bzw. Browsereinbettung, Website über die Bomberman gespielt werden kann & 2 \\ \hline
Computergegner mit variablem Schwierigkeitsgrad. & Abhängig von der Komplexität der KI \\ \hline
Weitere Gadgets oder Bomben. Denkbar wären Erweiterung der Reichweite, schnelleres Laufen, Fallen und so weiter &  1 Punkt für simple Erweiterung, mehr nach Aufwand. \\ \hline
Highscore / Punktesystem / Archivements, evntl. mit Onlinefunktion. & 2 für einfache Highscore, mehr nach Aufwand \\ \hline
Speichernd und Laden von Spielen. & 3 \\ \hline
Tutorial Level & 2 \\ \hline
\end{longtable}

Weitere Features nach Absprache mit den Tutoren.
\end{document}
