\documentclass{programmierpraktikum}
\vorlesung{Programmierpraktikum}
\semester{Sommersemester 2012}
\betreuer{Wilfried Linder}
\subtitle{Bomberman}

\begin{document}

\maketitle
Wie in Meilenstein 3 angegeben, müssen für das Bestehen zusätzliche Features implementiert werden. Diese können aus dem Katalog unten ausgewählt werden. Es ist ebenfalls möglich, eigene Features vorzuschlagen. Die Tutoren entscheiden dann von Fall zu Fall, wie viele Punkte vergeben werden.
\section{Erweiterungen Bomberman}
\begin{itemize}
  \item Weite Karten / Levels, Punkte nach Aufwand.
  \item Zufallsgenerator für Karten. Mehr Punkte werden vergeben, wenn der Generator z.B. Konsistenzprüfung beherrscht, also testen kann, ob ein Level gültig ist. Das heißt die Ausgänge können erreicht werden und kein Spieler ist bevorteilt. Punkte: 1 bzw. 3.
  \item Karteneditor, mit dem neue Levels angelegt werden können. Dabei sollen die Startpunkte der Spielfigur(en), Mauerstücke und Ausgänge und so weiter gesetzt werden können. Mehr Punkte werden vergeben, wenn der Editor z.B. Konsistenzprüfung beherrscht, also testen kann, ob ein Level gültig ist. Das heißt die Ausgänge können erreicht werden und kein Spieler ist bevorteilt. Punkte: 2 bzw. 5
  \item Sound. Ausgabe von Explosionseffekten, Hintergrundmusik, etc. Punkte: 1
  \item Applet bzw. Browsereinbettung, Website über die Bomberman gespielt werden kann. Punkte: 2
  \item Computergegner mit variablem Schwierigkeitsgrad. Punkte abhängig von der Komplexität der KI.
  \item Weitere Gadgets oder Bomben. Denkbar wären Erweiterung der Reichweite, schnelleres Laufen, Fallen und so weiter. 1 Punkt für simple Erweiterung, mehr nach Aufwand.
  \item Highscore / Punktesystem / Archivements, evntl. mit Onlinefunktion. Punkte: 2 für einfache Highscore, mehr nach Aufwand.
  \item Speichernd und Laden von Spielen. Punkte: 3
  \item Tutorial Level. Punkte: 2
\end{itemize}
Weitere Features nach Absprache mit den Tutoren.
\end{document}
